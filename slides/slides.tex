\documentclass[pdf,russian]{beamer}

\usepackage[T2A]{fontenc}
\usepackage[utf8]{inputenc}
\usepackage[russian]{babel}
\usepackage{booktabs}
\usepackage{minted}
\usepackage{tikz}

\usetikzlibrary{trees}

\selectlanguage{russian}

\mode<presentation>{
    \usetheme{CambridgeUS}
    \useoutertheme{infolines}
}

%% preamble
\title{Парсеры}
\subtitle{немного обо всём}
\author{Артем Оганджанян}
\institute{ЛОЛ-2017}
\date{21 августа 2017 г.}

\begin{document}

\AtBeginSection[]
{
   \begin{frame}
       \frametitle{Оглавление}
       \tableofcontents[currentsection]
   \end{frame}
}

\begin{frame}
    \titlepage
\end{frame}

\section{Калькулятор}
\subsection{Задача}

\begin{frame}[fragile]
    \frametitle{Калькулятор}
    \begin{block}{Задача}
        Command-line interface калькулятор.
    \end{block}

    \pause

    \begin{exampleblock}{Пример}
        \begin{semiverbatim}
\$ ./calculator
> 2 + 2
4
> 1 + 2 * (3 + 27 / 9)
13
> 123(321
syntax error
        \end{semiverbatim}
    \end{exampleblock}
\end{frame}

\begin{frame}
    \frametitle{Проблемы:}
    \pause
    \begin{itemize}
        \item скобки,
        \pause
        \item приоритет операций,
        \pause
        \item унарный минус,
        \pause
        \item линейное время работы\dots
    \end{itemize}
\end{frame}

\subsection{Лексемы}

\begin{frame}
    \frametitle{Лексемы}
    \begin{block}{Лексема}
        Набор символов, образующих синтаксическую единицу.
    \end{block}

    \pause

    \[
        \text{Input: \tt{`(21 + 8) * 2017'}}
    \]

    \pause

    \begin{table}
    \begin{tabular}{l @{\hspace{1cm}} rrrrrrr} \toprule
        Символы & \tt ( & \tt{21} & \tt + & \tt 8 & \tt ) & \tt *    & \tt{2017} \\
        Токены  &    $($& число   &    $+$& число &    $)$& $\times$ & число \\
        \bottomrule
    \end{tabular}
    \end{table}

    \pause

    \[
        \text{Output: [`$($', число, `$+$', число, `$)$', `$\times$', число]}
    \]
\end{frame}

\begin{frame}[fragile]
    \frametitle{Код}
    \pause
    \begin{minted}{kotlin}
enum class SimpleTokenType {
    ADD, SUB, MUL, DIV, L_PAREN, R_PAREN
}

sealed class Token
data class SimpleToken(val simpleTokenType: SimpleTokenType)
    : Token()
data class IntegerToken(val value: Int)
    : Token()
    \end{minted}
\end{frame}

\begin{frame}[fragile]
    \frametitle{Код}
    \begin{minted}{kotlin}
fun tokenize(str: String): List<Token> {
    var i = 0
    val tokens = ArrayList<Token>()

    fun skipSpaces() {
        while (i < str.length && str[i].isWhitespace()) {
            ++i
        }
    }
    \end{minted}
\end{frame}

\begin{frame}[fragile]
    \frametitle{Код}
    \begin{minted}{kotlin}
    skipSpaces()
    while (i < str.length) {
        fun consume(simpleTokenType: SimpleTokenType) {
            tokens.add(SimpleToken(simpleTokenType))
            ++i
        }

        when (str[i]) {
            '+' -> consume(SimpleTokenType.ADD)
            '-' -> consume(SimpleTokenType.SUB)
            '*' -> consume(SimpleTokenType.MUL)
            '/' -> consume(SimpleTokenType.DIV)
            '(' -> consume(SimpleTokenType.L_PAREN)
            ')' -> consume(SimpleTokenType.R_PAREN)
    \end{minted}
\end{frame}

\begin{frame}[fragile]
    \frametitle{Код}
    \begin{minted}{kotlin}
            in '0'..'9' -> {
                var value = 0
                while (i < str.length && str[i] in '0'..'9') {
                    value = value * 10 + (str[i] - '0')
                    ++i
                }
                tokens.add(IntegerToken(value))
            }
            else -> throw RuntimeException("error")
        }
        skipSpaces()
    }

    return tokens
}
    \end{minted}
\end{frame}

\subsection{Абстрактное синтаксическое дерево}

\begin{frame}
    \frametitle{Как понять, в каком порядке вычислять операции?}
    \begin{columns}
        \column{0.5\textwidth}
        \onslide<2->Имеем:
        \column{0.5\textwidth}
        \onslide<3->Хотим получить:
    \end{columns}
    \begin{columns}
        \column{0.5\textwidth}
        \onslide<2->
        \[
            (1 + 2) \times (3 + 27 / 9)
        \]

        \column{0.5\textwidth}
        \onslide<3->
        \center
        \begin{tikzpicture}[level 1/.style={sibling distance=5em},level 2/.style={sibling distance=3em}]]
            \node {$\times$}
            child {
                node {$+$}
                child { node {1} }
                child { node {2} }
            }
            child {
                node {$+$}
                child { node {3} }
                child {
                    node {$/$}
                    child { node {27} }
                    child { node {9} }
                }
            };
        \end{tikzpicture}
    \end{columns}
    \onslide<4->Такое дерево называется \emph{абстрактным синтаксическим деревом}.
\end{frame}

\begin{frame}
    \frametitle{Грамматика}
    \pause
    \begin{tabular}{lcl}
        $\langle$Выражение$\rangle$  & $::=$ & $\langle$Выражение$\rangle$ `$+$' $\langle$Слагаемое$\rangle$ \\
                                     & $ | $ & $\langle$Выражение$\rangle$ `$-$' $\langle$Слагаемое$\rangle$ \\
                                     & $ | $ & $\langle$Слагаемое$\rangle$ \\ \\
        \pause
        $\langle$Слагаемое$\rangle$  & $::=$ & $\langle$Слагаемое$\rangle$ `$\times$' $\langle$Умножаемое$\rangle$ \\
                                     & $ | $ & $\langle$Слагаемое$\rangle$ `$/$' $\langle$Умножаемое$\rangle$ \\
                                     & $ | $ & $\langle$Умножаемое$\rangle$ \\ \\
        \pause
        $\langle$Умножаемое$\rangle$ & $::=$ & Число \\
                                     & $ | $ & `$-$'$\langle$Умножаемое$\rangle$ \\
                                     & $ | $ & `$($'$\langle$Выражение$\rangle$`$)$'
    \end{tabular}
\end{frame}

\begin{frame}
    \frametitle{Дерево разбора}
    \begin{columns}
        \column{0.3\textwidth}
        \[
            (1 + 2) \times 6
        \]

        \column{0.7\textwidth}
        \center
        \begin{tikzpicture}[level distance=0.8cm,sibling distance=5em]
            \node {$\langle$Выражение$\rangle$}
            child {
                node {$\langle$Слагаемое$\rangle$}
                child {
                    node {$\langle$Слагаемое$\rangle$}
                    child {
                        node {$\langle$Умножаемое$\rangle$}
                        child { node {`$($'} }
                        child {
                            node {$\langle$Выражение$\rangle$}
                            child {
                                node {$\langle$Выражение$\rangle$}
                                child {
                                    node {$\langle$Слагаемое$\rangle$}
                                    child {
                                        node {$\langle$Умножаемое$\rangle$}
                                        child {
                                            node {Число}
                                        }
                                    }
                                }
                            }
                            child { node {`$+$'} }
                            child {
                                node {$\langle$Слагаемое$\rangle$}
                                child {
                                    node {$\langle$Умножаемое$\rangle$}
                                    child {
                                        node {Число}
                                    }
                                }
                            }
                        }
                        child { node {`$)$'} }
                    }
                }
                child { node {`$\times$'} }
                child {
                    node {$\langle$Умножаемое$\rangle$}
                    child { node {Число} }
                }
            };
        \end{tikzpicture}
    \end{columns}
\end{frame}

\begin{frame}
    \frametitle{Составление грамматики}
    Разбор выражения `$1-2-3$'.
    \center
    $\langle$Выражение$\rangle$ $::=$
    \begin{columns}
        \column{0.5\textwidth}
        \center
        $\langle$Выражение$\rangle$ `$-$' $\langle$Слагаемое$\rangle$
        \hspace{2cm}
        \begin{tikzpicture}
            \node {$-$}
            child {
                node {$-$}
                child { node {1} }
                child { node {2} }
            }
            child { node {3} };
        \end{tikzpicture}
        \[
            1-2-3=-4
        \]
        \column{0.5\textwidth}
        \center
        $\langle$Слагаемое$\rangle$ `$-$' $\langle$Выражение$\rangle$
        \hspace{2cm}
        \begin{tikzpicture}
            \node {$-$}
            child { node {1} }
            child {
                node {$-$}
                child { node {2} }
                child { node {3} }
            };
        \end{tikzpicture}
        \[
            1-2-3=1-(2-3)=1?
        \]
    \end{columns}
\end{frame}

\begin{frame}[fragile]
    \frametitle{Код}
    \pause
    \begin{minted}{kotlin}
class Parser(val tokens: List<Token>, var i: Int = 0) {
  fun parse(): Int {
    val result = parseExpression()
    if (i != tokens.size) {
      throw RuntimeException("incorrect expression")
    }
    return result
  }
    \end{minted}
\end{frame}

\begin{frame}[fragile]
    \frametitle{Код}
    \begin{minted}{kotlin}
  fun parseExpression(): Int {
    var result = parseTerm()
    while (i < tokens.size &&
        (tokens[i] == SimpleToken(SimpleTokenType.ADD) ||
         tokens[i] == SimpleToken(SimpleTokenType.SUB))) {
      if (tokens[i] == SimpleToken(SimpleTokenType.ADD)) {
          ++i
          result += parseTerm()
      } else {
          ++i
          result -= parseTerm()
      }
    }
    return result
  }
    \end{minted}
\end{frame}

\begin{frame}[fragile]
    \frametitle{Код}
    \begin{minted}{kotlin}
  fun parseTerm(): Int {
    var result = parsePrim()
    while (i < tokens.size &&
        (tokens[i] == SimpleToken(SimpleTokenType.MUL) ||
         tokens[i] == SimpleToken(SimpleTokenType.DIV))) {
      if (tokens[i] == SimpleToken(SimpleTokenType.MUL)) {
          ++i
          result *= parsePrim()
      } else {
          ++i
          result /= parsePrim()
      }
    }
    return result
  }
    \end{minted}
\end{frame}

\begin{frame}[fragile]
    \frametitle{Код}
    \begin{minted}{kotlin}
  fun parsePrim(): Int {
    if (i == tokens.size) {
        throw RuntimeException("incorrect expression")
    }
    val token = tokens[i]
    ++i
    \end{minted}
\end{frame}

\begin{frame}[fragile]
    \frametitle{Код}
    \begin{minted}{kotlin}
  return when (token) {
    is IntegerToken -> {
      token.value
    }
    SimpleToken(SimpleTokenType.SUB) -> parsePrim()
    SimpleToken(SimpleTokenType.L_PAREN) -> {
      val result = parseExpression()
      if (i == tokens.size ||
          tokens[i] != SimpleToken(SimpleTokenType.R_PAREN)) {
        throw RuntimeException("incorrect expression")
      }
      ++i
      result
    }
    else -> throw RuntimeException("incorrect expression")
  }
    \end{minted}
\end{frame}

\end{document}
